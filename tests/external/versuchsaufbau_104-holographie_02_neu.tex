%%%%%%%%%%%%%%%%%%%%%%%%%%%%%%%%%%%%%%%%%%%%%%%%%%%%%%%%%%%%%%%%%%%%%%%%%%%%%%%
% Title:       Versuchsaufbau Herstellung holographisches Gitter
% Versuch:     104 - Holographie
% Author:      Tobias Kr�hling
% Needs:       pstricks, pst-node, pst-optexp
% DesignScale: unit=10mm
% Size:		   13.9 x 3.2
% FontSize:    11pt
% Fonts:       pxfonts
%
%%%%%%%%%%%%%%%%%%%%%%%%%%%%%%%%%%%%%%%%%%%%%%%%%%%%%%%%%%%%%%%%%%%%%%%%%%%%%%%
\documentclass{scrartcl}
\usepackage[latin1]{inputenc}
\usepackage[T1]{fontenc}
\usepackage[ngerman]{babel}
\usepackage{lmodern}
\usepackage{pst-optexp}


\begin{document}
\begin{pspicture}[showgrid=true](-2.2,-1.4)(11.7,1.8)
   \newrgbcolor{lensColor}{0.87 0.91 0.93}
   \psset{labeloffset=1}
   \pnode(0,0){Laser}
   \pnode(1,0){Blende}
   \pnode(1.5,0){RaumfilterA}
   \pnode(3.5,0){RaumfilterE}
   \pnode(5,0){Linse}
   \pnode(5,0.2){LinseA}
   \pnode(5,-0.2){LinseB}
   \pnode(10,1){Film}
   \pnode(10,0.2){Spiegel1}
   \pnode(9.2,-0.3){Spiegel2}
   \pnode(9.28,-0.2){Spiegel2A}
   \psline[linewidth=2\pslinewidth,linecolor=red!90!black]
      (Laser)(Linse)
   \psline[linewidth=2\pslinewidth,linecolor=red!90!black]
      (LinseA)(Spiegel1)(Film)
   \psline[linewidth=2\pslinewidth,linecolor=red!90!black]
      (LinseB)(Spiegel2A)(Film)
   \optbox[endbox,optboxwidth=2,labeloffset=-1](Blende)(Laser){HeNe-Laser}
   \optbox[endbox,optboxwidth=2,labeloffset=-1](Linse)(RaumfilterE){Raumfilter}
   \pinhole(Laser)(RaumfilterA){Blende}
   \lens[endbox,fillstyle=solid,fillcolor=lensColor](RaumfilterE)(Linse){Linse}
   \mirror(Linse)(Spiegel1)(Film){}
   \rput[l](10.3,0){\small
      \rput[l](0,0.5em){Spiegel}
      \rput[l](0,-0.5em){(fest)}
   }
   \mirror[variable](Linse)(Spiegel2)(Film){}
   \rput(9.2,-1.05){\small var.~Spiegel}
   \rput(Film){
      \psframe(0.5,1\pslinewidth)(-0.5,5\pslinewidth)
      \psline[linewidth=2\pslinewidth](0.5,0)(-0.5,0)
      \rput(0,0.5){\small Film}
   }
\end{pspicture}

\begin{pspicture}[showgrid=true](-2.2,-1.4)(11.7,1.8)
   \newrgbcolor{lensColor}{0.87 0.91 0.93}
   \psset{labeloffset=1}
   \pnode(0,0){Laser}
   \pnode(10,1){Film}
   \pnode(Film|Laser){Ref}
   \optbox[endbox, optboxwidth=2](Ref)(Laser){HeNe-Laser}
   \pinhole[abspos=0.8](Laser)(Ref){Blende}
   \optbox[optboxwidth=2, abspos=2.5](Laser)(Ref){Raumfilter}
   \lens[fillstyle=solid, fillcolor=lensColor](Laser)(Ref){Linse}
   \mirror[labelalign=b]([offset=0.2]Laser)([offset=0.2]Ref)(Film){\begin{tabular}{c}Spiegel\\(fest)\end{tabular}}
   \opttripole[label=0.4 180](Ref)(Film)(Ref){%
     \psframe(0.5,1\pslinewidth)(-0.5,5\pslinewidth)
     \psline[linewidth=2\pslinewidth](0.5,0)(-0.5,0)
   }{Film}
   \addtopsstyle{Beam}{linewidth=2\pslinewidth, linecolor=red!90!black}
   \drawbeam{1}{2}{3}{4}.
   \drawbeam[startpos=0 0.2]{4}{5}{(Film)}.
   \mirror[variable, labeloffset=0.8]([offset=-0.2]Laser)([offset=-0.2, Xnodesep=-0.7]Ref)(Film){var. Spiegel}
   \drawbeam[startpos=0 -0.2]{4}{7}{(Film)}.
\end{pspicture}
\end{document}
%
% vim: set expandtab ai sw=3 ts=3:
