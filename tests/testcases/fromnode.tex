Test what happens if the input interface is missed (this situation could be
fixed if you use \textbackslash optplane instead of the nodes for drawing.

\begin{LTXexample}[pos=t]
\begin{pspicture}[showgrid=true](0.5,-0.3)(6,3.3)
  \psset{beaminsidelast}
  % a single beam misses the input interface
  \pnode(1,3){A}\pnode(1,1){G}\pnode(2,1){B}
  \beamsplitter(A)(G)(B)
  \drawbeam{(A)}{}
  % both beams hit the input interface because of beamdiv > 0
  \pnode(2,3){A}\pnode(2,1){G}\pnode(3,1){B}
  \beamsplitter(A)(G)(B)
  \drawwidebeam[beamdiv=4]{(A)}{}
  % the lower beam misses the input interface
  \pnode(3,3){A}\pnode(3,1){G}\pnode(4,1){B}
  \beamsplitter(A)(G)(B)
  \drawwidebeam[beamdiv=4, beamangle=2]{(A)}{}
  % the upper beam misses the input interface
  \pnode(4,3){A}\pnode(4,1){G}\pnode(5,1){B}
  \beamsplitter(A)(G)(B)
  \drawwidebeam[beamdiv=4, beamangle=-2]{(A)}{}
  % both beams miss the input interface
  \pnode(5,3){A}\pnode(5,1){G}\pnode(6,1){B}
  \beamsplitter(A)(G)(B)
  \drawwidebeam[beamwidth=0.1]{(A)}{}
\end{pspicture}
\end{LTXexample}


\begin{pspicture}[showgrid=true](6,5)
  \pnode(2,4){A}
  \optbox[endbox](A)(4,4)
  \optbox[endbox](A)(4,3)
  \optbox[endbox](A)(2,2)
  \drawbeam{(A)}{1}
  \drawbeam{(A)}{2}
  \drawbeam{(A)}{3}
\end{pspicture}
